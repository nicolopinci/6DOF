%% LyX 2.2.4 created this file.  For more info, see http://www.lyx.org/.
%% Do not edit unless you really know what you are doing.
\documentclass[english]{article}
\usepackage[T1]{fontenc}
\usepackage[latin9]{inputenc}
\usepackage{babel}
\begin{document}

\title{Report}
\maketitle

\section*{Equations, transformations and matrices}

In order to define the system, we have to consider the math behind
it. First of all, we have to define the transformations through matrices:
\begin{itemize}
\item Pitch motion matrix\\
$\left[\begin{array}{ccc}
1 & 0 & 0\\
0 & cos(\nu_{p}) & -sin(\nu_{p})\\
0 & sin(\nu_{p}) & cos(\nu_{p})
\end{array}\right]$\\
where $\nu_{p}$ represents the rotation angle around the x axis,
therefore it is the clockwise rotation in the yz plane
\item Roll motion matrix\\
$\left[\begin{array}{ccc}
cos(\nu_{r}) & 0 & -sin(\nu_{r})\\
0 & 1 & 0\\
sin(\nu_{r}) & 0 & cos(\nu_{r})
\end{array}\right]$\\
where $\nu_{r}$ represents the rotation angle around the y axis,
therefore it is the clockwise rotation in the xz plane
\item Heave motion consists in adding a constant to the z value after having
performed the pitch and roll motion matrices
\end{itemize}
Given three connection points $P_{1}$, \textbf{$P_{2}$ }and $P_{3}$,
they can be expressed as $[P_{1},P_{2},P_{3}]$. In this problem we
have:
\begin{itemize}
\item $P_{1}($
\end{itemize}

\end{document}
